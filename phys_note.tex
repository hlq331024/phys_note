\documentclass[12pt, oneside]{book}

\usepackage{fourier}  % font

\usepackage{amsmath}  % \begin{align}\end{align}
\usepackage{amssymb}  % \checkmark

\usepackage[mathcal]{euscript}

\usepackage{geometry}  % scale
\geometry{scale=0.70}

\begin{document}
\tableofcontents
	
\chapter{Time-dependent perturbation theory}
\begin{gather*}
1\rightsquigarrow2\rightsquigarrow1\\
1\rightsquigarrow2\rightarrow1\rightarrow2\rightarrow1\rightarrow2\rightsquigarrow1\\
0\rightarrow1\rightsquigarrow2\rightarrow0\rightarrow2\rightarrow0\rightarrow2\rightsquigarrow1\rightarrow0
\end{gather*}

\section*{The interaction picture}
Suppose that the Hamiltonian $H=H_0+H_1$, where $H_0$ is time-independent and $H_1$ is time-dependent. Suppose that we prepare or measure the system such that it's in an eigenstate of $H_0$, namely $H_0\psi_i=E_i\psi_i$. After some time $t$, due to the influence of $H_1$, the initial state $\psi_i$ will generally evolve to a superposition of eigenstates of $H_0$, and the transition probability to the state $\psi_j$ is
\[
p_{i\to j}=|\langle\psi_j|e^{-iHt}|\psi_i\rangle|^2
\]
and we have $\sum_jp_{i\to j}=1$ due to the completeness of the eigenstates $|\psi_j\rangle$.

Notice that Hamiltonian at different times generally do not commute, and in that case the operator $e^{-iHt}$ should really mean the time-ordered version of it. What we shall derive in the following should be true in the general case.

We would like to calculate the effect of $H_1$ on the transition probabilities $\{p_{i\to j}\}$ perturbatively -- in terms of a series in $H_1$. If we have $H_1=0$, then $p_{i\to j}=\delta_{ij}$. A method called the "interaction picture" simplifies the calculation somewhat. In effect, it does the following:
\[
p_{i\to j}=|\langle\psi_j|e^{-iHt}|\psi_i\rangle|^2=|\langle\psi_j|e^{iH_0t}e^{-iHt}|\psi_i\rangle|^2
\]
and the unitary operator $e^{iH_0t}e^{-iHt}$ satisfies the following differential equation
\begin{align*}
d_te^{iH_0t}e^{-iHt}
&=-ie^{iH_0t}H_1e^{-iHt}\\
&=-ie^{iH_0t}H_1e^{-iH_0t}e^{iH_0t}e^{-iHt}\\
&=:-iH_{1I}e^{iH_0t}e^{-iHt}
\end{align*}
with solution given by the Dyson series
\[
e^{iH_0t}e^{-iHt}=1-i\int_0^tdt'H_{1I}-\int_0^tdt'\int_0^{t'}dt''H_{1I}H_{1I}+\cdots
\]
and thus to first order in $H_1$, the transition amplitude (up to a phase) is
\begin{align*}
\langle\psi_j|e^{iH_0t}e^{-iHt}|\psi_i\rangle
&=\delta_{ij}-i\int_0^tdt'\langle\psi_j|H_{1I}|\psi_i\rangle\\
&=\delta_{ij}-i\int_0^tdt'e^{-i\omega_{ij}t'}\langle\psi_j|H_1|\psi_i\rangle
\end{align*}

The unitary operator $e^{iH_0t}e^{-iHt}$ can also be written as
\[
e^{iH_0t}e^{-iHt}=\mathcal{T}\exp\left(-i\int_0^tdt'H_{1I}\right)
\]
which may be generalized to
\[
U_{21}:=e^{iH_0t_2}e^{-iHt_{21}}e^{-iH_0t_1}=\mathcal{T}\exp\left(-i\int_{t_1}^{t_2}dt'H_{1I}\right)
\]

\section*{Case I: Constant perturbation}
Suppose that $H_1=V$ is a constant operator, then the transition amplitude above evaluates to
\[
\langle\psi_j|e^{iH_0t}e^{-iHt}|\psi_i\rangle=\delta_{ij}+V_{ji}\frac{e^{-i\omega_{ij}t}-1}{\omega_{ij}}
\]
and if $j\ne i$, the transition probability is
\[
p_{i\to j}=|V_{ji}|^2\left(\frac{\sin{\omega_{ij}t\mathbin/2}}{\omega_{ij}\mathbin/2}\right)^2
\]
which takes the form of the familiar "sinc" function. Interestingly, when $t$ is very large, the transition probability $p_{i\to j}$ has the limiting behavior
\[
\lim_{t\to\infty}p_{i\to j}=|V_{ji}|^2\,2\pi t\,\delta^1(\omega_{ij})
\]
This is a first sign of Fermi's golden rule.

However, the result above should only hold for sufficiently small $t$, since the expression for the transition probability does not always satisfy $p_{i\to j}\le1$. For example,
\[
\omega_{ij}=0\Rightarrow
p_{i\to j}=|V_{ji}|^2t^2
\]
and thus we require that $t\ll|V_{ji}|^{-1}$.

\section*{Case II: Monochromatic perturbation}
Suppose that $H_1=Ve^{-i\omega t}+V^{\dagger}e^{i\omega t}$, where $V$ is a constant operator, then the transition amplitude is equal to
\[
\langle\psi_j|e^{iH_0t}e^{-iHt}|\psi_i\rangle=\delta_{ij}+V_{ji}\frac{e^{-i\omega_{ij}t}e^{-i\omega t}-1}{\omega_{ij}+\omega}+V_{ij*}\frac{e^{-i\omega_{ij}t}e^{i\omega t}-1}{\omega_{ij}-\omega}
\]
which contains two terms if $j\ne i$, thus $p_{i\to j}$ will contain four terms if $j\ne i$.

In order to make some simplification, we may attempt to take $t\to\infty$. However, as we have remarked earlier, $t$ cannot be taken too large, otherwise perturbation theory will break down.

We may focus on the case when $\omega$ is close to $\omega_{ij}$ or $-\omega_{ij}$. If for example $\omega$ is close to $\omega_{ij}$, then the second term dominates, and the transition probability $p_{i\to j}$ when $j\ne i$ is approximately equal to
\[
p_{i\to j}\approx|V_{ij}|^2\left(\frac{\sin{\omega_{ij-}t\mathbin/2}}{\omega_{ij-}\mathbin/2}\right)^2
\]
where $\omega_{ij-}:=\omega_{ij}-\omega$.

Although we may not take $t\to\infty$, when the final state lies in a spectrum continuum, it can be justified (see Tong Chen's notes) that we may take $t$ large, such that
\[
\langle\psi_j|e^{iH_0t}e^{-iHt}|\psi_i\rangle\approx\delta_{ij}-iV_{ji}2\pi\delta^1(\omega_{ij}+\omega)-iV_{ij*}2\pi\delta^1(\omega_{ij}-\omega)
\]
which means that if $j\ne i$,
\[
p_{i\to j}\approx|V_{ji}|^2\,2\pi t\,\delta^1(\omega_{ij}+\omega)+|V_{ij}|^2\,2\pi t\,\delta^1(\omega_{ij}-\omega)
\]
since $\omega\ne0$, the cross terms vanishes. Here we have used one of the delta functions to evaluate the other delta function integral, and obtained $t$.

\section*{Einstein's coefficients}
\begin{align*}
d_tn_1&=-B_{21}\rho_{\omega}n_1+A_{12}n_2+B_{12}\rho_{\omega}n_2\\
d_tn_2&=-d_tn_1
\end{align*}
and Einstein: when $d_tn_1=d_tn_2=0$,
\[
\frac{n_2}{n_1}=\frac{B_{21}\rho_{\omega}}{A_{12}+B_{12}\rho_{\omega}}=e^{-\beta\omega}
\]

\section*{The electromagnetic field is quantized}
Let us suppose that
\[
H_1=Va+V^{\dagger}a^{\dagger}
\]
which gives us the transition probability
\begin{align*}
p_{1,n\to2,n-1}&=n_{\omega}\,|V_{21}|^2\,2\pi t\,\delta^1(\omega_{21}-\omega)\\
p_{2,n\to1,n+1}&=(n_{\omega}+1)\,|V_{21*}|^2\,2\pi t\,\delta^1(\omega_{21}-\omega)
\end{align*}
and thus
\begin{align*}
d_tn_1&\propto-n_{\omega}\,|V_{21}|^2n_1+(n_{\omega}+1)\,|V_{21}|^2n_2\\
d_tn_2&=-d_tn_1
\end{align*}

For phonons, the energy density is
\[
\frac{E}{V}=2\int\frac{d^3k}{(2\pi)^3}\frac{\omega}{e^{\beta\omega}-1}=\int\frac{k^2dk}{\pi^2}\frac{\omega}{e^{\beta\omega}-1}=\int\frac{\omega^2d\omega}{\pi^2}\frac{\omega}{e^{\beta\omega}-1}=\int\frac{\omega^3n_{\omega}d\omega}{\pi^2}
\]
which gives us the spectral energy density
\[
\rho_{\omega}=\frac{\omega^3n_{\omega}}{\pi^2}
\]
and we read
\begin{align*}
B_{21}&=B_{12}\\
A_{12}&=B_{12}\,\frac{\omega^3}{\pi^2}
\end{align*}

\chapter{The family of Green's functions}
\section*{Linear response theory}
We have
\begin{align*}
	\mathcal{O}
	&=e^{iHt}\mathcal{O}_Se^{-iHt}\\
	&=e^{iHt}e^{-iH_0t}e^{iH_0t}\mathcal{O}_Se^{-iH_0t}e^{iH_0t}e^{-iHt}\\
	&=e^{iHt}e^{-iH_0t}\mathcal{O}_Ie^{iH_0t}e^{-iHt}
\end{align*}
and we may expand $e^{iH_0t}e^{-iHt}$ to first order in $H_1$, and obtain
\[
e^{iHt}\mathcal{O}_Se^{-iHt}=\mathcal{O}_I-i\int_0^tdt'\mathcal{O}_IH_{1I}+i\int_0^tdt'H_{1I}\mathcal{O}_I
\]
which means that the linear response is characterized by the commutator
\[
\mathcal{O}_IH_{1I}-H_{1I}\mathcal{O}_I
\]
at different times: $\mathcal{O}_I$ is at time $t$ while $H_{1I}$ is at time $t'$, and we have $\int_0^tdt'$.
\begin{align*}
	e^{iHt_2}\mathcal{O}_{2S}e^{-iHt_2}
	&=e^{iH_0t_2}\mathcal{O}_{2S}e^{-iH_0t_2}-i\int_0^{t_2}dt_1e^{iH_0t_2}\mathcal{O}_{2S}e^{-iH_0t_{21}}H_{1S}e^{-iH_0t_1}\\
	&\qquad+i\int_0^{t_2}dt_1e^{iH_0t_1}H_{1S}e^{iH_0t_{21}}\mathcal{O}_{2S}e^{-iH_0t_2}
\end{align*}
which is easily proved by breaking the unitary evolution operator $e^{-iHt}$ into pieces and expanding. This result is very intuitive.

\section*{An example: The simple harmonic oscillator}
We have the response function at zero temperature
\[
D_{\beta\to\infty}=\frac{i}{2\omega}\,e^{-i\omega t_{21}}\Theta_{21}-\frac{i}{2\omega}\,e^{-i\omega t_{12}}\Theta_{21}
\]
and thus the linear response is
\[
\int dt_1\left(\frac{i}{2\omega}\,e^{-i\omega t_{21}}\Theta_{21}-\frac{i}{2\omega}\,e^{-i\omega t_{12}}\Theta_{21}\right)e^{0^+t_1}=\frac{i}{2\omega}\left(\frac1{i\omega}-\frac1{-i\omega}\right)=\frac1{\omega^2}
\]
which is correct, since the new equilibrium position is at $\frac{\lambda}{\omega^2}$ for $H_1=-\lambda\phi$.

\section*{The Lehmann representation}
1. The real-time Green's function.
\begin{align*}
iG_{\beta}
&=\Theta_{21}p_i\langle\psi_i|\phi_2\phi_1|\psi_i\rangle+\eta\Theta_{12}p_i\langle\psi_i|\phi_1\phi_2|\psi_i\rangle\\
&=\Theta_{21}p_i\langle\psi_i|\phi_2|\psi_j\rangle\langle\psi_j|\phi_1|\psi_i\rangle+\eta\Theta_{12}p_i\langle\psi_i|\phi_1|\psi_j\rangle\langle\psi_j|\phi_2|\psi_i\rangle\\
&=\Theta_{21}p_ie^{iE_{ij}t_{21}}(\phi_{2S})_{ij}(\phi_{1S})_{ji}+\eta\Theta_{12}p_ie^{iE_{ij}t_{12}}(\phi_{1S})_{ij}(\phi_{2S})_{ji}
\end{align*}
and we have
\[
\widetilde{G}_{\beta}=p_i\frac{(\phi_{2S})_{ij}(\phi_{1S})_{ji}}{\omega+E_{ij}+i0^+}-\eta p_i\frac{(\phi_{1S})_{ij}(\phi_{2S})_{ji}}{\omega-E_{ij}-i0^+}
\]

2.a. The retarded Green's function.
\begin{align*}
iD_{\beta+}
&=\Theta_{21}p_i\langle\psi_i|\phi_2\phi_1|\psi_i\rangle-\eta\Theta_{21}p_i\langle\psi_i|\phi_1\phi_2|\psi_i\rangle\\
&=\Theta_{21}p_i\langle\psi_i|\phi_2|\psi_j\rangle\langle\psi_j|\phi_1|\psi_i\rangle-\eta\Theta_{21}p_i\langle\psi_i|\phi_1|\psi_j\rangle\langle\psi_j|\phi_2|\psi_i\rangle\\
&=\Theta_{21}p_ie^{iE_{ij}t_{21}}(\phi_{2S})_{ij}(\phi_{1S})_{ji}-\eta\Theta_{21}p_ie^{iE_{ij}t_{12}}(\phi_{1S})_{ij}(\phi_{2S})_{ji}
\end{align*}
and we have
\[
\widetilde{D}_{\beta+}=p_i\frac{(\phi_{2S})_{ij}(\phi_{1S})_{ji}}{\omega+E_{ij}+i0^+}-\eta p_i\frac{(\phi_{1S})_{ij}(\phi_{2S})_{ji}}{\omega-E_{ij}+i0^+}
\]

2.b. The advanced Green's function.
\begin{align*}
-iD_{\beta-}
&=\Theta_{12}p_i\langle\psi_i|\phi_2\phi_1|\psi_i\rangle-\eta\Theta_{12}p_i\langle\psi_i|\phi_1\phi_2|\psi_i\rangle\\
&=\Theta_{12}p_i\langle\psi_i|\phi_2|\psi_j\rangle\langle\psi_j|\phi_1|\psi_i\rangle-\eta\Theta_{12}p_i\langle\psi_i|\phi_1|\psi_j\rangle\langle\psi_j|\phi_2|\psi_i\rangle\\
&=\Theta_{12}p_ie^{iE_{ij}t_{21}}(\phi_{2S})_{ij}(\phi_{1S})_{ji}-\eta\Theta_{12}p_ie^{iE_{ij}t_{12}}(\phi_{1S})_{ij}(\phi_{2S})_{ji}
\end{align*}
and we have
\[
\widetilde{D}_{\beta-}=p_i\frac{(\phi_{2S})_{ij}(\phi_{1S})_{ji}}{\omega+E_{ij}-i0^+}-\eta p_i\frac{(\phi_{1S})_{ij}(\phi_{2S})_{ji}}{\omega-E_{ij}-i0^+}
\]

3. The imaginary-time Green's function.
\begin{align*}
-\mathcal{G}_{\beta}
&=\Theta_{21}p_i\langle\psi_i|\phi_2\phi_1|\psi_i\rangle+\eta\Theta_{12}p_i\langle\psi_i|\phi_1\phi_2|\psi_i\rangle\\
&=\Theta_{21}p_i\langle\psi_i|\phi_2|\psi_j\rangle\langle\psi_j|\phi_1|\psi_i\rangle+\eta\Theta_{12}p_i\langle\psi_i|\phi_1|\psi_j\rangle\langle\psi_j|\phi_2|\psi_i\rangle\\
&=\Theta_{21}p_ie^{E_{ij}\tau_{21}}(\phi_{2S})_{ij}(\phi_{1S})_{ji}+\eta\Theta_{12}p_ie^{E_{ij}\tau_{12}}(\phi_{1S})_{ij}(\phi_{2S})_{ji}
\end{align*}
and we have, setting $t_1=0$ such that the $\Theta_{12}$ term does not contribute,
\[
\widetilde{\mathcal{G}}_{\beta}=p_i\frac{(\phi_{2S})_{ij}(\phi_{1S})_{ji}}{i\omega_l+E_{ij}}-\eta p_i\frac{(\phi_{1S})_{ij}(\phi_{2S})_{ji}}{i\omega_l-E_{ij}}
\]

\section*{The "master" Green's function}
is defined as
\[
\widetilde{\mathcal{M}}_{\beta}=p_i\frac{(\phi_{2S})_{ij}(\phi_{1S})_{ji}}{z+E_{ij}}-\eta p_i\frac{(\phi_{1S})_{ij}(\phi_{2S})_{ji}}{z-E_{ij}}
\]
or equivalently,
\[
\widetilde{\mathcal{M}}_{\beta}=\frac{(p_i-\eta p_j)(\phi_{2S})_{ij}(\phi_{1S})_{ji}}{z+E_{ij}}
\]

For a non-interacting Hamiltonian, when $\phi_{1S}=a^{\dagger}$ and $\phi_{2S}=a$, the master Green's function takes a very simple form:
\[
\widetilde{\mathcal{M}}_{\beta}=\frac1{z-\varepsilon}
\]
which is most easily proved by taking $|\psi_i\rangle$ to be the occupation-number basis.

\section*{The spectral function}
is defined as
\[
\widetilde{A}_{\beta}=i\widetilde{D}_{\beta+}-i\widetilde{D}_{\beta-}
\]
which is equal to
\[
p_i(\phi_{2S})_{ij}(\phi_{1S})_{ji}\,2\pi\delta^1(\omega+E_{ij})-\eta p_i(\phi_{1S})_{ij}(\phi_{2S})_{ji}\,2\pi\delta^1(\omega-E_{ij})
\]
or equivalently,
\[
(p_i-\eta p_j)(\phi_{2S})_{ij}(\phi_{1S})_{ji}\,2\pi\delta^1(\omega+E_{ij})
\]

We have the exact identity
\[
\int\frac{d\omega}{2\pi}\,\widetilde{A}_{\beta}=1
\]
for a general Hamiltonian, when $\phi_{1S}=a^{\dagger}$ and $\phi_{2S}=a$.

\section*{The Kramers-Kronig relations}
It can be easily verified that
\[
\widetilde{\mathcal{M}}_{\beta}=\int\frac{d\omega}{2\pi}\frac{\widetilde{A}_{\beta}}{z-\omega}
\]
thanks to the delta function. Now we have, if $\text{im}\,z>0$,
\begin{align*}
\widetilde{\mathcal{M}}_{\beta}
&=\int\frac{d\omega}{2\pi}\frac{\widetilde{A}_{\beta}}{z-\omega}\\
&=i\int\frac{d\omega}{2\pi}\frac{\widetilde{D}_{\beta+}-\widetilde{D}_{\beta-}}{z-\omega}=i\int_{C-}\frac{d\omega}{2\pi}\frac{\widetilde{D}_{\beta+}-\widetilde{D}_{\beta-}}{z-\omega}=i\int_{C-}\frac{d\omega}{2\pi}\frac{\widetilde{D}_{\beta+}}{z-\omega}=i\int\frac{d\omega}{2\pi}\frac{\widetilde{D}_{\beta+}}{z-\omega}
\end{align*}
which gives us
\begin{align*}
\widetilde{D}_{\beta+}&=i\int\frac{d\omega'}{2\pi}\frac{\widetilde{D}_{\beta+}}{\omega-\omega'+i0^+}\\
\Rightarrow
\widetilde{D}_{\beta+}&=i\mathcal{P}\int\frac{d\omega'}{\pi}\frac{\widetilde{D}_{\beta+}}{\omega-\omega'}
\end{align*}
or equivalently, the Kramers-Kronig relations,
\begin{align*}
\text{re}\,\widetilde{D}_{\beta+}&=-\mathcal{P}\int\frac{d\omega'}{\pi}\frac{\text{im}\,\widetilde{D}_{\beta+}}{\omega-\omega'}\\
\text{im}\,\widetilde{D}_{\beta+}&=\mathcal{P}\int\frac{d\omega'}{\pi}\frac{\text{re}\,\widetilde{D}_{\beta+}}{\omega-\omega'}
\end{align*}

\chapter{Scattering theory}
\section*{A useful calculation}
\begin{align*}
-i\int_{-\infty}^tdt'\langle\psi_j|H_{1I}|\psi_i\rangle
&=-i\int_{-\infty}^tdt'e^{-i\omega_{ij}t'}\langle\psi_j|H_1|\psi_i\rangle\\
&=-i\int_{-\infty}^tdt'e^{-i\omega_{ij}t'}e^{0^+t}\langle\psi_j|H_1|\psi_i\rangle\\
&=\frac1{\omega_{ij}+i0^+}\,e^{-i\omega_{ij}t'}e^{0^+t}\langle\psi_j|H_1|\psi_i\rangle\\
&=\frac1{\omega_{ij}+i0^+}\,e^{-i\omega_{ij}t'}\langle\psi_j|H_1|\psi_i\rangle
\end{align*}

\section*{The scattering $S$-matrix}
is defined by
\[
S:=U_{+\infty\,-\infty}
\]

We have, using the calculation above,
\[
S_{ji}=\langle\psi_j|U_{+\infty\,-\infty}|\psi_i\rangle=\delta_{ji}-iT_{ji}2\pi\delta^1(\omega_{ij})
\]
The transition probability
\[
p_{i\to j}=|T_{ji}|^2\,2\pi t\,\delta^1(\omega_{ij})
\]
where we have defined
\[
T=V+V\frac1{E_i-H_0+i0^+}V+V\frac1{E_i-H_0+i0^+}V\frac1{E_i-H_0+i0^+}V+\cdots
\]

\section*{The Lippmann-Schwinger equation}
We define
\begin{align*}
|\psi_{i+}\rangle&:=U_{0\,-\infty}|\psi_i\rangle\\
|\psi_{i-}\rangle&:=U_{0\,+\infty}|\psi_i\rangle
\end{align*}
or equivalently,
\begin{align*}
\lim_{t\to-\infty}e^{-iHt}|\psi_{i+}\rangle&=\lim_{t\to-\infty}e^{-iH_0t}|\psi_i\rangle\\
\lim_{t\to+\infty}e^{-iHt}|\psi_{i-}\rangle&=\lim_{t\to+\infty}e^{-iH_0t}|\psi_i\rangle
\end{align*}
and by definition,
\begin{align*}
S_{ji}
&=\langle\psi_j|U_{+\infty\,-\infty}|\psi_i\rangle\\
&=\langle\psi_j|U_{+\infty\,0}U_{0\,-\infty}|\psi_i\rangle\\
&=\langle\psi_{j-}|\psi_{i+}\rangle
\end{align*}

To obtain the expression of $|\psi_{i+}\rangle$, we compute using the "useful calculation",
\[
\langle\psi_j|\psi_{i+}\rangle=\langle\psi_j|U_{0\,-\infty}|\psi_i\rangle=\langle\psi_j|\psi_i\rangle+\langle\psi_j|\frac1{E_i-H_0+i0^+}T|\psi_i\rangle
\]
from which we read the Lippmann-Schwinger equation
\begin{align*}
|\psi_{i+}\rangle
&=|\psi_i\rangle+\frac1{E_i-H_0+i0^+}T|\psi_i\rangle\\
&=|\psi_i\rangle+\frac1{E_i-H_0+i0^+}V|\psi_{i+}\rangle
\end{align*}

Since by definition $H_0|\psi_i\rangle=E_i|\psi_i\rangle$, we have
\[
H_0|\psi_{i+}\rangle+V|\psi_{i+}\rangle=E_i|\psi_{i+}\rangle
\]
namely $|\psi_{i+}\rangle$ is an eigenstate of the total $H$, with the same eigenvalue $E_i$. Therefore, if $|\psi_i\rangle$ is a scattering eigenstate of $H_0$, then $|\psi_{i+}\rangle$ is a scattering eigenstate of $H$.

\section*{Scattering problem in three dimensions}
The scattering problem in three dimensions is a good place to show the usefulness of the Lippmann-Schwinger equation. To set up the problem, notice that the scattering eigenstates of the free Hamiltonian $H_0$ are the plane waves, and thus we may take the state $|\psi_i\rangle$ to be the plane wave with wave vector $k_i$.

We then use the Lippmann-Schwinger equation to calculate the scattering eigenstates of the total Hamiltonian
\[
|\psi_{i+}\rangle=|\psi_i\rangle+\frac1{E_i-H_0+i0^+}T|\psi_i\rangle
\]

Now let us compute the coordinate space wave function of the scattering eigenstates, and focus on its asymptotic behavior. Noticing the following calculation,
\[
\langle x|\frac1{E_i-H_0+i0^+}|y\rangle=2m\int\frac{d^3l}{(2\pi)^3}\frac{e^{il\cdot x}e^{-il\cdot y}}{k^2-l^2+i0^+}=-\frac{m}{2\pi}\frac{e^{ik_i\eta}}{\eta}
\]
where $\eta$ is the distance between $x$ and $y$. In the limit $x\to\infty$, we have
\[
\frac{e^{ik_i\eta}}{\eta}=\frac{e^{ik_ir}}{r}\,e^{-ik_iy\cdot\hat{x}}=\frac{e^{ik_ir}}{r}\,\langle\psi_j|y\rangle
\]
where $|\psi_j\rangle$ is the plane wave with wave vector $k_j=k_i\hat{x}$. Thus we read
\[
\langle x|\psi_{i+}\rangle=\langle x|\psi_i\rangle-\frac{m}{2\pi}\frac{e^{ik_ir}}{r}\,\langle\psi_j|T|\psi_i\rangle
\]
which means that asymptotically, the coordinate space wave function of the scattering eigenstates of the total $H$ is the superposition of the incoming plane wave $|\psi_i\rangle$ and an outgoing spherical wave, modulated by an angular distribution function $-\frac{m}{2\pi}T_{ji}$. Since we already know that the transition probability is proportional to $|T_{ji}|^2$, we now know that it is, equivalently, proportional to the square of the angular distribution function.

\section*{The Wigner-Eckart theorem}
The statement of the Wigner-Eckart theorem is quite intuitive: if some operators $T_{jm}$ transform like the $|jm\rangle$ states, where $j$ is fixed, then we have
\[
\langle\alpha,jm|T_{j_1m_1}|\alpha_2,j_2m_2\rangle=\langle jm|j_1m_1,j_2m_2\rangle\langle\alpha,j\|T_{j_1}\|\alpha_2,j_2\rangle
\]
where $\langle\alpha,j\|T_{j_1}\|\alpha_2,j_2\rangle$ is a number that does not depend on $m$, $m_1$ or $m_2$.

\section*{Rotational symmetry}
In the special case that the scattering potential $V$ is rotationally invariant, the general non-perturbative analysis above can be carried out further. In this case, the total $H$ is rotationally invariant, and thus the scattering $S$- and $T$-matrices are rotationally invariant. In other words, they are scalar operators. The Wigner-Eckart theorem:
\begin{align*}
\langle E,lm|S|E',l'm'\rangle
&=\delta_{ll'}\delta_{mm'}\langle E,l\|S\|E',l'\rangle\\
&=\delta_{ll'}\delta_{mm'}\langle E,l\|S\|E',l\rangle
\end{align*}
Since we also know that the result should be proportional to $\delta_{EE'}$, we arrive at the conclusion that the scattering $S$-matrix is diagonal in the $|E,lm\rangle$ basis, with diagonal elements depending only on $E$ and $l$,
\[
\langle E,lm|S|E',l'm'\rangle=\delta_{ll'}\delta_{mm'}\delta_{EE'}S_{E,l}
\]
and due to unitarity, the diagonal element $S_{E,l}$ can only be a phase.

The $T$-matrix is also a scalar operator, and we have, considering the case $E'=E$,
\begin{align*}
\langle E,lm|T|E,l'm'\rangle
&=\delta_{ll'}\delta_{mm'}\langle E,l\|T\|E,l'\rangle\\
&=\delta_{ll'}\delta_{mm'}\langle E,l\|T\|E,l\rangle\\
&=\delta_{ll'}\delta_{mm'}T_{E,l}
\end{align*}
Thus we conclude that
\[
S_{E,l}=1-2\pi iT_{E,l}
\]

\section*{Spherical harmonics}
\[
\mathcal{D}_{\phi\theta\psi}|jm\rangle=\mathcal{D}_{\phi\theta\psi,jm'jm}|jm'\rangle
\]
We have
\begin{align*}
Y_{lm,\hat{n}*}
&=\langle lm|\hat{n}\rangle\\
&=\langle lm|\mathcal{D}_{\phi\theta\psi}|\hat{z}\rangle\\
&=\langle lm|\mathcal{D}_{\phi\theta\psi}|l'm'\rangle\langle l'm'|\hat{z}\rangle\\
&=\langle lm|\mathcal{D}_{\phi\theta\psi}|lm'\rangle\langle lm'|\hat{z}\rangle\\
&=\langle lm|\mathcal{D}_{\phi\theta\psi}|l0\rangle\langle l0|\hat{z}\rangle\\
&=\mathcal{D}_{\phi\theta\psi,lml0}Y_{l0,\hat{z}*}
\end{align*}
where $Y_{l0,\hat{z}}=\sqrt{\frac{2l+1}{4\pi}}$. This calculation shows that the spherical harmonics are related to the matrix representation of the rotation operators in a simple way. Notice that
\[
\mathcal{D}_{\phi\theta\psi,lml0}=\mathcal{D}_{\phi\theta0,lml0}
\]

We also have
\begin{align*}
\langle E,lm|p\rangle
&=\langle E,lm|\mathcal{D}_{\phi\theta\psi}|p\hat{z}\rangle\\
&=\langle E,lm|\mathcal{D}_{\phi\theta\psi}|E',l'm'\rangle\langle E',l'm'|p\hat{z}\rangle\\
&=\langle E,lm|\mathcal{D}_{\phi\theta\psi}|E,lm'\rangle\langle E,lm'|p\hat{z}\rangle\\
&=\langle E,lm|\mathcal{D}_{\phi\theta\psi}|E,l0\rangle\langle E,l0|p\hat{z}\rangle\\
&=\mathcal{D}_{\phi\theta\psi,lml0}\langle E,l0|p\hat{z}\rangle
\end{align*}

\chapter{Quantum field theory}
\section*{Gell-Mann and Low theorem}
We have, using $\rightsquigarrow$ and $\rightarrow$ diagrams, for example,
\begin{align*}
\phi_3\phi_2\phi_1
&=U_{03}\phi_{3I}U_{30}U_{02}\phi_{2I}U_{20}U_{01}\phi_{1I}U_{10}\\
&=U_{03}\phi_{3I}U_{32}\phi_{2I}U_{21}\phi_{1I}U_{10}
\end{align*}
where $\phi_j$ are time-dependent Heisenberg operators. We also have
\begin{align*}
U_{0\,-\infty}|\Omega_0\rangle&=\lim_{t_-\nwarrow-\infty}e^{iEt_-}e^{-iE_0t_-}|\Omega\rangle\langle\Omega|\Omega_0\rangle\\
\langle\Omega_0|U_{+\infty\,0}&=\lim_{t_+\searrow+\infty}e^{iE_0t_+}e^{-iEt_+}\langle\Omega_0|\Omega\rangle\langle\Omega|
\end{align*}
Notice that $U_{0\,-\infty}$ and $U_{+\infty\,0}$ are not unitary! since $t_-$ and $t_+$ are not real.
\[
\langle\Omega|\phi_3\phi_2\phi_1|\Omega\rangle=\frac{\langle\Omega_0|U_{+\infty\,3}\phi_{3I}U_{32}\phi_{2I}U_{21}\phi_{1I}U_{1\,-\infty}|\Omega_0\rangle}{\langle\Omega_0|U_{+\infty\,-\infty}|\Omega_0\rangle}
\]

\section*{The path-integral}
We have
\[
\langle\Omega|\phi_3\phi_2\phi_1|\Omega\rangle=\frac{\langle\Omega_0|U_{+\infty\,0}\phi_3\phi_2\phi_1U_{0\,-\infty}|\Omega_0\rangle}{\langle\Omega_0|U_{+\infty\,0}U_{0\,-\infty}|\Omega_0\rangle}
\]
after canceling some factors,
\[
\langle\Omega|\phi_3\phi_2\phi_1|\Omega\rangle=\lim_{t_-\nwarrow-\infty}\lim_{t_+\searrow+\infty}\frac{\langle\Omega_0|e^{-iHt_+}\phi_3\phi_2\phi_1e^{iHt_-}|\Omega_0\rangle}{\langle\Omega_0|e^{-iHt_+}e^{iHt_-}|\Omega_0\rangle}
\]

In fact, we have
\[
\langle\Omega|\phi_3\phi_2\phi_1|\Omega\rangle=\lim_{t_-\nwarrow-\infty}\lim_{t_+\searrow+\infty}\frac{\langle\Omega'|e^{-iHt_+}\phi_3\phi_2\phi_1e^{iHt_-}|\Omega''\rangle}{\langle\Omega'|e^{-iHt_+}e^{iHt_-}|\Omega''\rangle}
\]
for any $|\Omega'\rangle$ and $|\Omega''\rangle$. This is because
\[
\lim_{t\searrow+\infty}e^{-iHt}=\lim_{t\searrow+\infty}e^{-iEt}|\Omega\rangle\langle\Omega|
\]

\section*{Second quantization}
\[
\Psi=\Psi^0+\Psi^1+\Psi^2+\Psi^3+\cdots
\]
or more precisely,
\[
\mathcal{F}=\mathcal{H}^0\oplus\mathcal{H}^1\oplus\mathcal{H}^2\oplus\mathcal{H}^3\oplus\cdots
\]

We have
\[
\Psi=\Psi^0+C_i|\psi_i\rangle+C_{ij}|\psi_i\rangle|\psi_j\rangle+C_{ijk}|\psi_i\rangle|\psi_j\rangle|\psi_k\rangle+\cdots
\]
and
\begin{align*}
\mathcal{P}_{12}\Psi
&=\Psi^0+C_i|\psi_i\rangle+C_{ij}|\psi_j\rangle|\psi_i\rangle+C_{ijk}|\psi_j\rangle|\psi_i\rangle|\psi_k\rangle+\cdots\\
&=\Psi^0+C_i|\psi_i\rangle+C_{ji}|\psi_i\rangle|\psi_j\rangle+C_{jik}|\psi_i\rangle|\psi_j\rangle|\psi_k\rangle+\cdots
\end{align*}
which means that
\[
C_{ji\cdots}=\zeta C_{ij\cdots}
\]

Therefore, we define
\[
|\psi_i\psi_j\cdots\rangle=\frac1{\sqrt{n!}}\sum_{\mathcal{P}}\zeta_{\mathcal{P}}|\psi_{\mathcal{P}i}\rangle|\psi_{\mathcal{P}j}\rangle\cdots
\]
which is a over-complete basis of the $n$-particle subspace $\mathcal{H}^n$. Notice that
\begin{align*}
\langle\phi_i\phi_j\cdots|\psi_k\psi_l\cdots\rangle
&=\frac1{n!}\sum_{\mathcal{P}}\sum_{\mathcal{Q}}\zeta_{\mathcal{P}}\zeta_{\mathcal{Q}}\langle\phi_{\mathcal{P}i}|\psi_{\mathcal{Q}k}\rangle\langle\phi_{\mathcal{P}j}|\psi_{\mathcal{Q}l}\rangle\cdots\\
&=\sum_{\mathcal{P}}\zeta_{\mathcal{P}}\langle\phi_i|\psi_{\mathcal{P}k}\rangle\langle\phi_j|\psi_{\mathcal{P}l}\rangle\cdots\\
&=\det{\begin{pmatrix}
\langle\phi_i|\psi_k\rangle&\langle\phi_i|\psi_l\rangle&\cdots\\
\langle\phi_j|\psi_k\rangle&\langle\phi_j|\psi_l\rangle&\cdots\\
\vdots&\vdots&\ddots
\end{pmatrix}}
\end{align*}
which, in particular, means that
\[
\langle\psi_i\psi_i\cdots\psi_i\psi_j\psi_j\cdots\psi_j\cdots|\psi_i\psi_i\cdots\psi_i\psi_j\psi_j\cdots\psi_j\cdots\rangle=n_i!n_j!\cdots
\]

A natural complete orthonormal basis of the $n$-particle subspace $\mathcal{H}^n$ is the occupation number basis
\[
|\cdots n_i\cdots n_j\cdots\rangle=\frac{|\cdots\psi_i\psi_i\cdots\psi_i\cdots\psi_j\psi_j\cdots\psi_j\cdots\rangle}{\sqrt{\cdots n_i!\cdots n_j!\cdots}}
\]

\section*{Creation and annihilation operators}
We define
\[
(a_{\psi})^{\dagger}|\psi_i\psi_j\cdots\rangle:=|\psi\psi_i\psi_j\cdots\rangle
\]
such that using
\begin{align*}
\langle\phi_i\phi_j\cdots|a_{\psi}|\psi_k\psi_l\cdots\rangle
&=\langle\psi_k\psi_l\cdots|(a_{\psi})^{\dagger}|\phi_i\phi_j\cdots\rangle^*\\
&=\langle\psi_k\psi_l\cdots|\psi\phi_i\phi_j\cdots\rangle^*\\
&=\langle\psi\phi_i\phi_j\cdots|\psi_k\psi_l\cdots\rangle
\end{align*}
one can prove that
\[
(a_{\phi})^{\dagger}(a_{\psi})^{\dagger}-\zeta(a_{\psi})^{\dagger}(a_{\phi})^{\dagger}=0
\]
as well as
\[
\boxed{a_{\phi}(a_{\psi})^{\dagger}-\zeta(a_{\psi})^{\dagger}a_{\phi}=\langle\phi|\psi\rangle}
\]

\section*{Second quantized operators}
In the Fock space, one-body operators take the form
\[
\widetilde{\mathcal{O}}_1=\widetilde{\mathcal{O}}_1|_{\mathcal{H}0}\oplus\widetilde{\mathcal{O}}_1|_{\mathcal{H}1}\oplus\widetilde{\mathcal{O}}_1|_{\mathcal{H}2}\oplus\widetilde{\mathcal{O}}_1|_{\mathcal{H}3}\oplus\cdots
\]
where $\widetilde{\mathcal{O}}_1|_{\mathcal{H}n}$ is like the total angular momentum of $n$ particles
\[
\widetilde{\mathcal{O}}_1|_{\mathcal{H}n}=\big(\mathcal{O}_1\otimes1\otimes\cdots\otimes1\big)+\big(1\otimes\mathcal{O}_1\otimes\cdots\otimes1\big)+\cdots+\big(1\otimes1\otimes\cdots\otimes\mathcal{O}_1\big)
\]

It is easy to verify that
\begin{align*}
\widetilde{\mathcal{O}}_1
&=\mathcal{O}_{1i}(a_{\psi i})^{\dagger}a_{\psi i}\\
&=\mathcal{O}_{1i}\langle\phi_j|\psi_i\rangle\langle\psi_i|\phi_k\rangle(a_j)^{\dagger}a_k\\
&=\langle\phi_j|\mathcal{O}_1|\phi_k\rangle(a_j)^{\dagger}a_k
\end{align*}

For two-body operators
\[
\widetilde{\mathcal{O}}_2=\widetilde{\mathcal{O}}_2|_{\mathcal{H}0}\oplus\widetilde{\mathcal{O}}_2|_{\mathcal{H}1}\oplus\widetilde{\mathcal{O}}_2|_{\mathcal{H}2}\oplus\widetilde{\mathcal{O}}_2|_{\mathcal{H}3}\oplus\cdots
\]
Notice that we may always write
\[
\mathcal{O}_2=\sum_{I\le J}C_{IJ}\,\big[\big(\mathcal{O}_{1I}\otimes\mathcal{O}_{1J}\big)+\big(\mathcal{O}_{1J}\otimes\mathcal{O}_{1I}\big)\big]
\]
since the two-body operator is always symmetric under permutation.

For concreteness, take $\mathcal{H}^3$ as an example, we have
\begin{align*}
\widetilde{\mathcal{O}}_2|_{\mathcal{H}3}
&=C_{IJ}\,\big[\big(\mathcal{O}_{1I}\otimes\mathcal{O}_{1J}\otimes1\big)+\big(\mathcal{O}_{1I}\otimes1\otimes\mathcal{O}_{1J}\big)+\big(1\otimes\mathcal{O}_{1I}\otimes\mathcal{O}_{1J}\big)\big]\\
&\qquad+C_{IJ}\,\big[\big(\mathcal{O}_{1J}\otimes\mathcal{O}_{1I}\otimes1\big)+\big(\mathcal{O}_{1J}\otimes1\otimes\mathcal{O}_{1I}\big)+\big(1\otimes\mathcal{O}_{1J}\otimes\mathcal{O}_{1I}\big)\big]
\end{align*}
where we have suppressed the $\sum_{I\le J}$ symbol. Now this can be written as
\[
\widetilde{\mathcal{O}}_2|_{\mathcal{H}3}=C_{IJ}\,\widetilde{\mathcal{O}}_{1I}|_{\mathcal{H}3}\widetilde{\mathcal{O}}_{1J}|_{\mathcal{H}3}-C_{IJ}\,\widetilde{\mathcal{O}}_{1I}\widetilde{\mathcal{O}}_{1J}|_{\mathcal{H}3}
\]
where
\[
\widetilde{\mathcal{O}}_{1I}\widetilde{\mathcal{O}}_{1J}|_{\mathcal{H}3}=\big(\mathcal{O}_{1I}\mathcal{O}_{1J}\otimes1\otimes1\big)+\big(1\otimes\mathcal{O}_{1I}\mathcal{O}_{1J}\otimes1\big)+\big(1\otimes1\otimes\mathcal{O}_{1I}\mathcal{O}_{1J}\big)
\]

The two-particle operator can now be written as
\begin{align*}
\widetilde{\mathcal{O}}_2
&=C_{IJ}\,\langle\phi_i|\mathcal{O}_{1I}|\phi_j\rangle\langle\phi_k|\mathcal{O}_{1J}|\phi_l\rangle(a_i)^{\dagger}a_j(a_k)^{\dagger}a_l-C_{IJ}\,\langle\phi_i|\mathcal{O}_{1I}\mathcal{O}_{1J}|\phi_l\rangle(a_i)^{\dagger}a_l\\
&=C_{IJ}\,\langle\phi_i|\mathcal{O}_{1I}|\phi_j\rangle\langle\phi_k|\mathcal{O}_{1J}|\phi_l\rangle(a_i)^{\dagger}a_j(a_k)^{\dagger}a_l-C_{IJ}\,\langle\phi_i|\mathcal{O}_{1I}|\phi_j\rangle\langle\phi_k|\mathcal{O}_{1J}|\phi_l\rangle(a_i)^{\dagger}a_l\delta_{jk}\\
&=C_{IJ}\,\langle\phi_i|\mathcal{O}_{1I}|\phi_j\rangle\langle\phi_k|\mathcal{O}_{1J}|\phi_l\rangle(a_i)^{\dagger}(a_k)^{\dagger}a_la_j\\
&=C_{IJ}\,\langle\phi_k|\mathcal{O}_{1J}|\phi_l\rangle\langle\phi_i|\mathcal{O}_{1I}|\phi_j\rangle(a_k)^{\dagger}(a_i)^{\dagger}a_ja_l
\end{align*}
and we finally conclude that
\[
\widetilde{\mathcal{O}}_2=\frac12\,\langle\phi_i|\langle\phi_k|\mathcal{O}_2|\phi_j\rangle|\phi_l\rangle(a_i)^{\dagger}(a_k)^{\dagger}a_la_j
\]

\section*{The number operator}
is the identity operator $1:\mathcal{H}^1\to\mathcal{H}^1$ lifted to the Fock space: $\mathcal{N}=(a_j)^{\dagger}a_j$.

\section*{Quantum statistical mechanics}
is governed by
\[
\rho=Z^{-1}e^{-\beta\mathcal{H}+\beta\mu\mathcal{N}}
\]

\section*{The Legendre transformation}
We have
\[
Z=e^{iW}=\int D\phi\,e^{iS+iJ\cdot\phi}
\]
such that the expectation value
\[
\phi_{\text{exp}}=\frac1{iZ}\frac{\partial Z}{\partial J}=\frac{\partial W}{\partial J}
\]

The effective action $\Gamma$ is defined by the following Legendre transformation
\[
-\Gamma=J\cdot\phi-W
\]
and should be viewed as a function of $\phi$,
\[
-\delta\Gamma=\delta J\cdot\phi+J\cdot\delta\phi-\frac{\partial W}{\partial J}\cdot\delta J=J\cdot\delta\phi
\]

\section*{The effective action}
We define a path-integral using the effective action in place of the action
\[
Z_{\Gamma}=\int D\phi\,e^{i\Gamma+iJ\cdot\phi}
\]

We may use the saddle point approximation, which requires that
\[
\frac{\partial\Gamma}{\partial\phi}+J=0
\]
and we have
\[
Z_{\Gamma}\approx e^{i\Gamma+iJ\cdot\phi}=e^{iW}
\]
which means that the original $Z$ and $W$ can be computed by drawing tree diagrams of the effective action only! We thus conclude that the propagator and interaction vertices in the expression of the effective action are exact.

\section*{The quantum variational principle}
Finally, notice that at $J=0$, we have
\[
\delta\Gamma=-J\cdot\delta\phi=0
\]
and the field is now equal to
\[
\frac{\partial W}{\partial J}|_{J=0}=\phi_{\text{exp}}|_{J=0}
\]
namely $\phi_{\text{exp}}|_{J=0}$ is the solution to the equation $\delta\Gamma=0$. This is to be compared with the fact that $\phi_{\text{cl}}$ is the solution to the equation $\delta S=0$.

\section*{Conformal transformations}
A map $\phi:\big(M_1,g_1\big)\to\big(M_2,g_2\big)$ is called conformal if
\[
\phi^*g_2=\Omega^2g_1
\]
or equivalently in local coordinates,
\[
g_{2\mu\nu}\frac{\partial x^{2\mu}\circ\phi}{\partial x^{1\rho}}\frac{\partial x^{2\nu}\circ\phi}{\partial x^{1\sigma}}=\Omega^2g_{1\rho\sigma}
\]

\section*{Noether's theorem}
For the shift $\psi'=\psi+\delta\psi$, where $\delta\psi$ may depend on $\psi$,
\begin{align*}
\delta\mathcal{L}
&=\frac{\partial\mathcal{L}}{\partial\psi}\,\delta\psi+\frac{\partial\mathcal{L}}{\partial\partial_{\mu}\psi}\,\delta\partial_{\mu}\psi\\
&=\partial_{\mu}\left(\frac{\partial\mathcal{L}}{\partial\partial_{\mu}\psi}\right)\delta\psi+\frac{\partial\mathcal{L}}{\partial\partial_{\mu}\psi}\,\partial_{\mu}\delta\psi+\frac{\delta S}{\delta\psi}\,\delta\psi\\
&=\partial_{\mu}\left(\frac{\partial\mathcal{L}}{\partial\partial_{\mu}\psi}\,\delta\psi\right)+\frac{\delta S}{\delta\psi}\,\delta\psi
\end{align*}
namely some current is conserved \textbf{on-shell}.

\section*{The Ward identity}
For the shift $\phi'=\phi+\delta\phi$, where $\delta\phi$ may depend on $\phi$,
\begin{align*}
\int D\phi\,e^{-S}X
&=\int D\phi'\,e^{-S'}X'\\
&=\int D\phi\,e^{-S}\left(1-\int d^2x\,T_{\mu\nu}\partial_{\mu}\varepsilon_{\nu}\right)\big(X+\delta X\big)\\[10pt]
0
&=\int D\phi\,e^{-S}\left(\delta X-X\int d^2x\,T_{\mu\nu}\partial_{\mu}\varepsilon_{\nu}\right)\\
&=\int D\phi\,e^{-S}\left(\delta X+X\int d^2x\,\varepsilon_{\nu}\partial_{\mu}T_{\mu\nu}\right)
\end{align*}
Integrating the Ward identity over a small pillbox shows that $\int dx\,T_{0\nu}$ is the generator of spacetime symmetries.

\section*{The conformal Ward identity}
The right-hand-side of the equation above can actually be written as
\begin{align*}
0
&=\int D\phi\,e^{-S}\left(\delta X+X\int_{\mathcal{B}}d^2x\,\varepsilon_{\nu}\partial_{\mu}T_{\mu\nu}\right)\\
&=\int D\phi\,e^{-S}\left(\delta X-X\int_{\mathcal{B}}d^2x\,T_{\mu\nu}\partial_{\mu}\varepsilon_{\nu}+X\int_{\partial\mathcal{B}}\varepsilon_{\mu\rho}dx_{\rho}\,\varepsilon_{\nu}T_{\mu\nu}\right)\\[10pt]
0
&=\int D\phi\,e^{-S}\left(\delta'X+X\int_{\partial\mathcal{B}}\varepsilon_{\mu\rho}dx_{\rho}\,\varepsilon_{\nu}T_{\mu\nu}\right)
\end{align*}

\chapter{The Kondo problem}
\section*{The effective Hamiltonian method}
We have, since $\mathcal{P}_j\mathcal{P}_j=1$,
\[
\mathcal{P}_iH\mathcal{P}_j\mathcal{P}_j\psi=E\mathcal{P}_i\psi
\]
and when $\mathcal{P}_1+\mathcal{P}_2=1$,
\begin{align*}
H_{11}\psi_1+H_{12}\psi_2&=E\psi_1\\
H_{21}\psi_1+H_{22}\psi_2&=E\psi_2
\end{align*}

The second equation can be used to solve for $\psi_2$,
\[
\psi_2=\frac1{E-H_{22}}H_{21}\psi_1
\]
which gives us a self-consistent equation
\[
\left(H_{11}+H_{12}\frac1{E-H_{22}}H_{21}\right)\psi_1=E\psi_1
\]

\section*{Applications to degenerate perturbation theory}
To second order in the perturbation,
\begin{align*}
H_{\text{eff}}
&=E_11+V_{11}+V_{12}\frac1{E-H_{0\,22}-V_{22}}V_{21}\\
&\approx E_11+V_{11}+V_{12}\frac1{E_1-H_{0\,22}}V_{21}
\end{align*}

\section*{AFM coupling in the Hubbard model}
\[
H=-t\sum_{i\ne j}(a_{is})^{\dagger}a_{js}+U\sum_in_{i+}n_{i-}
\]
The effective Hamiltonian is equal to
\begin{align*}
H_{\text{eff}}
&=\frac{t^2}{E-U}\sum_{i\ne j}\sum_{k\ne l}(a_{is})^{\dagger}a_{js}(a_{kr})^{\dagger}a_{lr}\\
&=\frac{t^2}{E-U}\sum_{i\ne j}(a_{is})^{\dagger}a_{js}(a_{jr})^{\dagger}a_{ir}
\end{align*}
Notice that
\begin{align*}
\sum_{i\ne j}S_i\cdot S_j
&=\frac12\sum_{i\ne j}(a_{is})^{\dagger}(a_{jr})^{\dagger}a_{js}a_{ir}-\frac14\sum_{i\ne j}(a_{is})^{\dagger}(a_{jr})^{\dagger}a_{jr}a_{is}\\
&=\frac12\sum_{i\ne j}(a_{is})^{\dagger}a_{is}-\frac12\sum_{i\ne j}(a_{is})^{\dagger}a_{js}(a_{jr})^{\dagger}a_{ir}-\frac14\sum_{i\ne j}(a_{is})^{\dagger}a_{is}(a_{jr})^{\dagger}a_{jr}\\
&=\frac12-\frac12\sum_{i\ne j}(a_{is})^{\dagger}a_{js}(a_{jr})^{\dagger}a_{ir}
\end{align*}
which means that the effective Hamiltonian contains an interaction
\[
H_{\text{eff}}\approx\frac{2t^2}{U}\sum_{i\ne j}S_i\cdot S_j+\frac{t^2}{-U}
\]

\section*{The Anderson impurity model}
\[
H=\sum_{ks}\varepsilon_kn_{ks}+\sum_{ks}V_k(a_{ks})^{\dagger}A_s+\text{h.c.}+\sum_s\mathcal{E}N_s+UN_+N_-
\]
Notice that we have
\[
H_{02}=H_{20}=0
\]

\section*{The low-energy effective Hamiltonian}
In the low-energy subspace, we have
\begin{align*}
H_{10}\frac1{E-H_{00}}H_{01}
&=\sum_{ks}\sum_{lr}V_{k*}V_l(A_s)^{\dagger}a_{ks}\frac1{E-H_{00}}(a_{lr})^{\dagger}A_r\\
&=\sum_{ks}\sum_{lr}V_{k*}V_l(A_s)^{\dagger}a_{ks}(a_{lr})^{\dagger}A_r\frac1{E-H_{00}-\varepsilon_l}\\
&\approx\sum_{ks}\sum_{lr}V_{k*}V_l(A_s)^{\dagger}a_{ks}(a_{lr})^{\dagger}A_r\frac1{\mathcal{E}-\varepsilon_l}
\end{align*}
as well as
\begin{align*}
H_{12}\frac1{E-H_{22}}H_{21}
&=\sum_{ks}\sum_{lr}V_kV_{l*}(a_{ks})^{\dagger}A_s\frac1{E-H_{22}}(A_r)^{\dagger}a_{lr}\\
&=\sum_{ks}\sum_{lr}V_kV_{l*}(a_{ks})^{\dagger}A_s(A_r)^{\dagger}a_{lr}\frac1{E-H_{22}+\varepsilon_l}\\
&\approx\sum_{ks}\sum_{lr}V_kV_{l*}(a_{ks})^{\dagger}A_s(A_r)^{\dagger}a_{lr}\frac1{-\mathcal{E}-U+\varepsilon_l}
\end{align*}
and thus
\[
H_{\text{l-e}}\approx H_{11}-\sum_{ks}\sum_{lr}V_{k*}V_l\left[\frac{(A_s)^{\dagger}a_{ks}(a_{lr})^{\dagger}A_r}{-\mathcal{E}+\varepsilon_l}+\frac{(a_{lr})^{\dagger}A_r(A_s)^{\dagger}a_{ks}}{\mathcal{E}+U-\varepsilon_k}\right]
\]

\section*{The Kondo model}
Notice that
\begin{align*}
s_{lk}\cdot S
&=\frac12\sum_{sr}(a_{lr})^{\dagger}(A_s)^{\dagger}A_ra_{ks}-\frac14\sum_{sr}(a_{lr})^{\dagger}(A_s)^{\dagger}A_sa_{kr}\\
&=\frac12\sum_{sr}(a_{lr})^{\dagger}(A_s)^{\dagger}A_ra_{ks}-\frac14\sum_r(a_{lr})^{\dagger}a_{kr}
\end{align*}
we see that the low-energy effective Hamiltonian contains an interaction
\[
2\sum_{kl}V_{k*}V_l\left[\frac1{-\mathcal{E}+\varepsilon_l}+\frac1{\mathcal{E}+U-\varepsilon_k}\right]s_{lk}\cdot S
\]
which is anti-ferromagnetic near the Fermi surface.

\chapter{Miscellaneous}
\section*{The guiding center coordinates}
are conserved, namely they commute with the Hamiltonian, and we have
\[
\big[X,Y\big]=\big[x+l^2\pi_y,y-l^2\pi_x\big]=-il^2
\]
Notice that they travel on equipotential lines
\begin{align*}
i\dot{X}_1&=\big[X_1,V\big]=-il^2\partial_2V\\
i\dot{X}_2&=\big[X_2,V\big]=+il^2\partial_1V
\end{align*}

\section*{Spin-waves and magnons}
The Schwinger-boson representation
\begin{align*}
S_+&=a^{\dagger}b\\
S_-&=b^{\dagger}a\\
2S_z&=a^{\dagger}a-b^{\dagger}b
\end{align*}
gives us the Holstein-Primakoff transformation if we use the constraint
\[
a^{\dagger}a+b^{\dagger}b=2S
\]
The spin-wave is obtained by, e.g., expanding around $S_z=-S$, namely setting
\[
b=b^{\dagger}=\sqrt{2S-a^{\dagger}a}\approx\sqrt{2S}
\]

\section*{The Bogoliubov transformation}
In order to diagonalize a Hamiltonian of the form
\[
\Psi_{i\dagger}A_{ij}\Psi_j
\]
while keeping the commutation relations, we want
\[
U^{\dagger}AU=\Lambda
\]
where $\Lambda$ is a diagonal matrix, as well as
\[
C_{ij}=\big[\Psi_i,\Psi_{j\dagger}\big]=\big[U_{ik}\Phi_k,U_{jl*}\Phi_{l\dagger}\big]=U_{ik}U_{jl*}C_{kl}\Rightarrow
C=UCU^{\dagger}
\]
where $C$ is, by assumption, a diagonal matrix. Therefore we have
\[
CAU=UCU^{\dagger}AU=UC\Lambda
\]
which means that $U$ diagonalizes the matrix $CA$, with eigenvalues $C\Lambda$.

\section*{The Wilsonian renormalization group}
1. The "effective action".
\[
Z=\int D\phi_{\Lambda}\exp{iS_{\Lambda}}=\int D\phi_{\Lambda'}\exp{iS_{\Lambda'}}=\cdots
\]

2. Effective field theory.
\[
\text{$q_{\Lambda}=\frac{Q_{\Lambda}}{\Lambda^{\dim{Q}}}$ is meaningful}
\]

3. Renormalization group fix points.
\[
\text{$\big\{q_{\Lambda'}\big\}=\big\{q_{\Lambda}\big\}$ up to a field rescaling}
\]
\end{document}